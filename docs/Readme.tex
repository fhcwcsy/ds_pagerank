\documentclass[12pt]{article}

%Packages BEGIN
\usepackage{amsmath}
\usepackage{physics}
\usepackage[margin=0.9in]{geometry}
\usepackage{listings}
\usepackage{xcolor}
%Packages END

\definecolor{codegreen}{rgb}{0,0.6,0}
\definecolor{codegray}{rgb}{0.5,0.5,0.5}
\definecolor{codepurple}{rgb}{0.58,0,0.82}
\definecolor{backcolour}{rgb}{0.95,0.95,0.92}

\lstdefinestyle{mystyle}{
    backgroundcolor=\color{backcolour},   
    commentstyle=\color{codegreen},
    keywordstyle=\color{magenta},
    numberstyle=\tiny\color{codegray},
    stringstyle=\color{codepurple},
    basicstyle=\ttfamily\footnotesize,
    breakatwhitespace=false,         
    breaklines=true,                 
    captionpos=b,                    
    keepspaces=true,                 
    numbers=left,                    
    numbersep=5pt,                  
    showspaces=false,                
    showstringspaces=false,
    showtabs=false,                  
    tabsize=2
}

\lstset{style=mystyle} 


\title{Data Structures PA1 Report}
\author{{\scriptsize B07202020}\\Hao-Chien Wang}

\begin{document}
\maketitle

\section{Usage}%
Use \texttt{./bin/pagerank <DIFF> <d>} to run the program. The input files are
placed under \texttt{./input/} and the output files are generated under the main
folder (\texttt{./}). The scripts are placed under \texttt{./src/} and can be 
compiled using \texttt{make}. You can also run the script \texttt{./runall.sh}
to run for all input combinations. Use \texttt{make clean} to clean output files,
executables and \texttt{*.o} files.

\section{Data Structures and Algorithms}%

I wrote my program in c++. I defined 2 classes: \texttt{Page} and 
\texttt{FindPageRank}. The foremer stores the index of the page (for example,
the index of \texttt{page499} is 499, saved as \texttt{int}), the number of 
outbranching links (\texttt{int}), pages it links to (\texttt{vector<int>}, storing
indices of the pages) and its pagerank (\texttt{double}). The latter reads
the input, compute pageranks and output the requested files. It stores the following
information:
\begin{enumerate}
	\item Both input values \texttt{DIFF} and \texttt{d} (as \texttt{double})
	\item A wordlist (\texttt{map<string, set<int>*}) with the string being all
		the words mentioned in the pages and the set (that the pointer points 
		to) saving the indices of the pages that mentioned the word. 
	\item An array of pages, saving all the \texttt{Pages} objects.
\end{enumerate}
All the above are defined in \texttt{src/pagerank.h} and \texttt{src/pagerank.cpp}. 
The output is generated in the following steps:

\begin{enumerate}
	\item Read input, create \texttt{Page} objects, generate wordlist and
		\texttt{reverseindex.txt} (done by the constructor of \texttt{FindPageRank}).
	\item Iterate until the terminating conditions are met (done by 
		\texttt{FindPageRank.iterate()}).
	\item Output \texttt{pr\_xx\_yyy.txt} (done by \texttt{FindPageRank.printPR()}).
	\item Read \texttt{list.txt}, search in wordlist according to it and output 
		\texttt{result\_xx\_yyy.txt} (done by \texttt{FindPageRank.search()}).
\end{enumerate}

\section{Complexity Analysis}%
The space complexity is $O\left(n^2\right)$ since we need to save all the pages,
and the size of each pages is $O(n)$ (assuming sorting does not use memory exceeding
$O\left(n^2\right)$). To analyze time complexity, we first examine
each steps:

\begin{enumerate}
	\item Reading input\\
	Reading the outbranching links takes $O\left(n^2\right)$ since we need to
	read all the input files and save them in our data structures. Reading
	the words mentioned in each pages is obviously $O(nw)$, where $w$ is the words
	mentioned by each pages. Writing \texttt{reverseindex.txt} also takes
	$O(nw)$. Therefore the time complexity of this step is $O\left(n^2+nw\right)$.
		
	\item Iteration\\
	In this step we compute the expression:
	\begin{equation*}
		M(M(M \cdots (M\vec{v}_0 + \vec{v}_1) + \vec{v}_1 ) + \vec{v}_1) \cdots + \vec{v} _1 
		= M^n \vec{v}_0 + \left(M^{n-1} + M^{n-2} + \cdots+ M + I\right) \vec{v}_1 
	\end{equation*}
	This expression converges exponentially fast (note that the second term is 
	a geometric series). Therefore the time complexity of this step is
	$O\left(-\log d - \log \text{DIFF}\right)$

	\item Print pagerank outputs.\\
	The time complexity of this step is obviously $O(n)$.

	\item Searching\\

\end{enumerate}

\end{document}
